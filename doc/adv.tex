\documentstyle[11pt]{article}

\pagestyle{empty}

\begin{document}

\centerline{\huge\sf Demos Commander}
\vskip 6pt
\centerline{\sf Copyright \copyright\ 1988--1990 Serge Vakulenko}
\vskip 12pt

DECO is a screen interface for the UNIX operating system (and compatible)
and is simple enough to be used by a beginner, as well as by an experienced
programmer. To simplify exploring and using Demos Commander (DECO),
you should keep it in mind, that it is analogous with famous
``Norton Commander''. DECO makes possible the following
operations:
\begin{itemize}
\item
Displaying user's name, system's name, terminal data, date and current
time in the continuous mode, what is of importance while
operating on the networks.
\item
Moving along the UNIX file tree.
\item
Displaying simultaneously one or two directories on the screen,
selecting files in them, copying, deleting, linking, or renaming files,
comparing the directories.
\item
Executing any UNIX commands, using names of
selected files to construct them, if necessary.
\item
Viewing and editing files, using built-in, as well
as user's own programs.
\item
Repeating commands, using the history.
\item
Setting windows' format, number of file attributes displayed, method of
viewing and editing files and other system parameters;
saving setup information in a file.
\item
Designing user's menus and calling for programs, with the menu system.
\end{itemize}

DECO can be used with any UNIX terminal, attached to UNIX operating system.
The terminal must have direct cursor addressing mode and 80-character screen
width.

DECO uses all possibilities of your terminal---keypad and functional keys,
colors, line drawing character set.

\end{document}
