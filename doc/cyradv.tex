\documentstyle[11pt]{article}

\pagestyle{empty}

\begin{document}

\centerline{\huge\sf Демос Командер}
\vskip 6pt
\centerline{\sf Copyright \copyright\ 1988--1990 Деиос/$\star$}
\vskip 12pt

DECO~--- экранная оболочка для операционных систем семейства Unix,
построенная по принципу наглядности и рассчитанная на
работу как неквалифицированного пользователя, так и программиста.
По интерфейсу DECO во многом похож на хорошо известный ``Norton Commander''.
DECO позволяет:

\begin{itemize}
\item
Отображать постоянно информацию об имени пользователя, имени системы,
терминале, дату и время, что важно при работе в сетях ЭВМ.
\item
Осуществлять наглядное движение по дереву файлов ОС Unix.
\item
Отображать одновременно на экране терминала один или два справочника,
отмечать в них файлы, копировать, удалять, связывать или переименовывать
файлы, сравнивать справочники.
\item
Выполнять любые команды OC Unix, используя для их конструирования при
необходимости имена отмеченных файлов.
\item
Просматривать и редактировать файлы, используя как встроенные, так и
стандартные программы.
\item
Повторять команды, используя ``историю работы''.
\item
Задавать формат окон, степень подробности информации о файлах,
способ просмотра и редактирования файлов и другие атрибуты системы;
запоминать настройку в файле.
\item
Создавать пользовательские меню и вызывать программы через систему
меню.
\end{itemize}

DECO может работать практически на любом терминале, подключенном к ОС Unix.
Основные требования к терминалу~--- наличие прямой адресации курсора
и ширина строки в 80 символов.  Низкая скорость терминальной линии~---
не помеха. DECO, работая в экранном режиме, минимизирует поток
данных, выдаваемых на терминал.

Кроме этого, DECO позволяет использовать практически все возможности Вашего
терминала: функциональную клавиатуру, цвета, набор символов псевдографики.

\end{document}
