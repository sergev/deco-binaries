\documentstyle[11pt,twoside]{guide}

\input{psfig}                   % load postscript macros

\pagestyle{headings}

\tolerance=1000

\newbox\tempbox                 % temporary box

\newcommand\vs{\char '040 }     % A visual space character
\newcommand\bs{\char '134 }     % A backslash character for \tt font
\newcommand{\lb}{\char '173 }   % A left brace character for \tt font
\newcommand{\rb}{\char '175 }   % A right brace character for \tt font

\newcommand{\fig}[3]{
  \begin{figure}[bp]
  \centerline{\psfig{figure=pic/#1.ps}}
  \caption{#3}
  \label{#2}
  \end{figure}
}

\newenvironment{example}{
  \begin{list}{}{
    \setlength{\leftmargin}{1.1in}
    \setlength{\labelwidth}{1in}
    \setlength{\labelsep}{0.1in}
    \setlength{\itemindent}{0in}
  }
  \newcommand{\litem}[1]{\item[##1\hfill]       % left flushed item
    \setbox\tempbox=\hbox{\makelabel{##1}~~}
    \ifdim \wd\tempbox > \labelwidth
      \par{\hfil}\par
    \fi
  }
  \small
}{
  \normalsize
  \end{list}
}

\newenvironment{source}{
  \small
  \begin{quote}
}{
  \end{quote}
  \normalsize
}

\begin{document}

\renewcommand{\thepage}{\roman{page}}

\tableofcontents
\listoftables
\listoffigures
\newpage

\setcounter{page}{1}
\renewcommand{\thepage}{\arabic{page}}


\section{Introduction}

\subsection{Brief Product Description}

DECO is a screen interface for the UNIX operating system (and compatible)
and is simple enough to be used by a beginner, as well as by an experienced
programmer. To simplify exploring and using Demos Commander (DECO),
you should keep it in mind, that it is analogous with famous
``Norton Commander''. DECO makes possible the following
operations:
\begin{itemize}
\item
Displaying user's name, system's name, terminal data, date and current
time in the continuous mode, what is of importance while
operating on the networks.
\item
Moving along the UNIX file tree.
\item
Displaying simultaneously one or two directories on the screen,
selecting files in them, copying, deleting, linking, or renaming files,
comparing the directories.
\item
Executing any UNIX commands, using names of
selected files to construct them, if necessary.
\item
Viewing and editing files, using built-in, as well
as user's own programs.
\item
Repeating commands, using the history.
\item
Setting windows' format, number of file attributes displayed, method of
viewing and editing files and other system parameters;
saving setup information in a file.
\item
Designing user's menus and calling for programs, with the menu system.
\end{itemize}

DECO can be used with any UNIX terminal, if this 
terminal is supplied with direct cursor addressing, 
upper and lowercase and is described in the system.
It is desirable to have a terminal with 10 functional 
keys. Such keys, as {\tt <PgUp>}, {\tt <PgDn>},
{\tt <Ins>}, {\tt <Del>}, and
analogous ones make the operation more simple (see 
also ``Learning the Keyboard'').

So far as UNIX standard terminal descriptions have
neither a description of functional keys nor color 
setting, nor line drawing characters, DECO uses an extended
(in comparison with the commonly used one) terminal 
description in {\em termcap} format. DECO has its own
file with additions for the standard termcap file;
this file can be used together with the system 
description. For more detail about the extended 
termcap version, see ``DECO Release Notes''.

\subsection{Conventional Signs}

Conventional signs used in the present guide are as 
follows:
\begin{itemize}
\item
In the keyboard description, {\tt \^{}X} means {\tt <Ctrl>~X},
{\tt FX} means the functional key~X, {\tt <Enter>} means
``Enter'' or ``Return'' keys. (See section ``Learning the Keyboard'').
\item
All pictures are printed in black and white.
\end{itemize}

By default, all examples are to be used by a user 
named ``alex'' on a host computer named ``berta'' in a
directory named {\tt /usr/user} which includes {\tt doc} and {\tt w}
subdirectories, a few files in each subdirectory.

\fig{p1}{screen}{DECO Screen}

\section{History of Development}

The language of mutual operation of the user and the 
system is one of the factors which influence the 
conditions of using the computer.

From the very start of their activities, a group of
programmers who were engaged in creating 
applications for Soviet computers, running under UNIX-compatible
operating systems, developed some screen
interfaces for UNIX. A great part of the products has 
been used by the authors themselves for personal 
needs, some screen interfaces, however, found their 
way to the hearts of end users. In contrast to system 
programmers, users are rather interested in the 
simplicity and a visual way of working of the system, 
and they are ready to forgive certain insufficiency and
even (in some cases) inefficiency of the system. With 
the appearance of personal computers,
demands of users increased, so screen interfaces 
available in UNIX ceased to suit them.

While developing DECO, authors proceeded from the fact, that a significant
majority of UNIX end users are experienced in using personal computers,
and their most popular command interface
is Norton Commander. Most users are satisfied
with its capabilities, in spite of some disadvantages, and they don't
want to learn again, while using UNIX. The Norton 
Commander interface is reused everywhere, where it is 
possible, so that users could use DECO without additional training.

Operation conditions on personal computers and under UNIX from
a terminal are, naturally, rather different. Not only the terminal
and the keyboard are different, but the standard operational situation (for
example, full length of filename and dimensions of file systems), and
the operating system's capabilities (for example, linking of files
under UNIX), and needs of users as well. So, even those DECO commands that
have analogy with Norton Commander ones do not always work in the same way.

\section{Basic Concepts}

\subsection{Starting DECO}

Suppose that a user named ``alex'' operates a host
computer named ``berta'' and the user wishes to operate
the {\tt w} directory. To run DECO, type:

\begin{source}
\begin{verbatim}
deco w
\end{verbatim}
\end{source}

The screen is shown in fig.~\ref{screen}.
This display will be discussed later. The main point 
now is to learn operating the keyboard. The fact is that 
DECO is developed to operate with the functional 
keyboard, while it depends strongly on the type of the 
terminal.

We shall start by calling for help. If you use a console of PC, just press
{\tt F1}. If you use VT--100, VT--200, VT--300 or compatible terminals,
press~{\tt -1-} on the left of the digital keyboard. For some terminals
with functional keys, help is started by {\tt F1}, quite similar to
the PC console. If you had no success, it may be caused either by
improper DECO installing or by setting up the system
for your terminal. Even in this case, DECO may be 
used, if the display looks like fig.~\ref{screen}. Press {\tt <Escape>} and
then digit 1 on the main keyboard. For some terminals, instead of
{\tt <Escape>} you should press {\tt \^{}[} ({\tt <Ctrl>} and
symbol {\tt [} simultaneously). If in this case
nothing happens, or the initial display does not appear,
see instruction on installation of terminals on your operating system.
Most probably your terminal is not described in the system or its type is
not, or incorrectly, indicated.

\fig{p2}{help}{Help}

When everything is done correctly, the display will be 
as shown in fig.~\ref{help}.
This is just a hint. It is meant for those, who have an 
idea about DECO commands, but forgot how to enter 
them. Commands are discussed in detail below and 
symbols for keys have already been shown above:~{\tt \^{}X}
means {\tt <Ctrl>}+X.

To exit the hint, just press any key. 

There is another way to obtain a helpful information, even more
complete than by the {\tt F1} key: just press {\tt F9} and enter
{\em main menu}. This menu describes both commands that
are executable through this menu, and their execution 
by means of control-characters (when it is possible). The
main menu is described in detail below (chapter ``Extended Functions'').

\subsection{Learning the Keyboard}

In contrast to many screen Shells, DECO is intended for 
using facilities of the terminal keyboard in the best 
possible way. In this guide, all commands are named just as on
the PC AT keyboard. On other terminals they may be designated
somewhat differently.

Under DECO, any command key has either its direct
analogue in any other terminal, or is replaced by 
pressing two keys successively: {\tt <Escape>} and character.
On your terminal some functional keys may be absent,
so you should remember the correspondence of the keys.
As a rule, keys of similar names are being used. For most terminals,
{\tt -1-}$\ldots$~{\tt -9-}, {\tt -0-} of the additional digital keyboard
(which is placed on the right of the terminal keyboard) are used
instead of {\tt F1}$\ldots${\tt F10}, respectively.

Correspondence of keys is shown in table~\ref{keyboard}.

\begin{table}[bp]
\begin{center}
\begin{tabular}{|l|l|l|}
\hline
Key & Key & Alternative \\
PC AT & VT--220 & input \\
\hline
Enter & Return & \^{}M \\
F1 & -1- & Escape 1 \\
F2 & -2- & Escape 2 \\
F3 & -3- & Escape 3 \\
F4 & -4- & Escape 4 \\
F5 & -5- & Escape 5 \\
F6 & -6- & Escape 6 \\
F7 & -7- & Escape 7 \\
F8 & -8- & Escape 8 \\
F9 & -9- & Escape 9 \\
F10 & -0- & Escape 0 \\
$\uparrow$ & up & Escape u \\
$\downarrow$ & down & Escape d \\
$\leftarrow$ & left & Escape l \\
$\rightarrow$ & right & Escape r \\
Home & Home & Escape h \\
End & End & Escape e \\
PgUp & PrevPage & Escape p \\
PgDn & NextPage & Escape n \\
Ins & Ins & \^{}T \\
Del & Del & \^{}G \\
\hline
\end{tabular}
\caption{Keyboard Usage}
\label{keyboard}
\end{center}
\end{table}

\subsection{Exit from DECO and from DECO Commands}

To quit DECO, press {\tt F10} or its analogue in the
terminal: {\tt -0-}~---the right bottom key in VT--220 or VT--100,
{\tt <Escape>}~0 if there is no such key at all. You may also
press {\tt <Ctrl>~C} ({\tt \^{}C}). The prompt appears (fig.~\ref{quit}).
 
\fig{p3}{quit}{Quit from DECO}

The display shows three items.
You have to select the answer and press
{\tt <Enter>}. To select the answer, use the arrow keys ``$\leftarrow$'' and
``$\rightarrow$'' or {\tt <Tab>} to change the current answer:
\begin{example}
\litem{Yes}
To quit DECO. The user goes back to the directory from which DECO was started.
\litem{No}
To continue the work.
\litem{Exec shell}
Command interpreter is started instead of DECO ({\tt /bin/sh} by default),
the user stays in the current directory.
\end{example}

{\tt F10} (or~{\tt \^{}C}) has a more general meaning. It {\em aborts} the
current command. If the user has started DECO
menu, or built-in command (viewing or editing files), 
{\tt F10} interrupts it and resets DECO. The only exception
is the non-screen DECO operation mode (it takes place 
after executing UNIX commands, see below). For
example, if you press {\tt F10} twice in the usual mode, a
question about quitting DECO will appear after your 
first pressing, and by the second pressing you will 
cancel the question and restore the screen.

Another way to exit DECO is to enter the built-in command {\tt exit}.
Type ``exit'' in the command line and press {\tt <Enter>}.
It stops DECO immediately.

Many users are accustomed to using {\tt <Escape>}, to
quit programs. For DECO, single pressing on {\tt <Escape>} in
any menu or in a built-in program makes the same 
effect as {\tt F10}, but with a two-second delay, to distinguish
the single {\tt <Escape>} from the beginning of code sequence
which is generated by functional keys.

If you don't want to wait two seconds, press {\tt <Escape>} twice.
It is similar to pressing {\tt F10}.

\subsection{How to Restore the Screen}

UNIX is a multitask system. It may happen, that, while
DECO is under operation, some other task brings a 
message onto the screen, or the terminal breaks up, or 
a message comes from the system (for example, a 
message about an approaching work cessation). To 
restore the terminal screen, always use the~{\tt \^{}]}
command. This command erases the screen and redraws the display.

\subsection{What is Shown on the Screen}

After starting DECO, one of possible displays is on the screen
(fig.~\ref{screen}). There are four screen formats and
a number of output modes, which are selected by 
DECO commands. These commands are discussed below.

The uppermost line includes data on the system, terminal and user.
This information is necessary, especially, while operating on a network.
Two panels are placed below (right and left), which 
correspond to two directories. One panel is current, its 
name is inverted or painted, the second panel is used 
in commands of copying or comparing, to give information about the
current directory, etc. By default the current panel is the left one.

Files of the corresponding directory are shown in the 
panels. The left symbol means the file type.
All possible file types are listed in table~\ref{filetypes}.

\begin{table}[bp]
\begin{center}
\begin{tabular}{|cl|}
\hline
Character & File type \\
\hline
space & regular file \\
{\tt *} & executable file \\
{\tt /} & directory \\
{\tt \&} & directory without search permission \\
{\tt \$} & character special file \\
{\tt \#} & block special file \\
{\tt =} & named pipe (fifo) \\
{\tt @} & symbolic link \\
{\tt \~{}} & symbolic link without file referred \\
{\tt !} & socket \\
\hline
\end{tabular}
\caption{File Types}
\label{filetypes}
\end{center}
\end{table}

If the file name has an extension (the right part of the name with
the point), it is separated from the name and
moved to the right---it is supposed to be a suffix which
denotes the file destination. Files are put in order by 
their types, suffixes, and names (in alphabetical order).

One of the files in the current panel is inverted or 
painted---this is the current file. The name and some
attributes of the current file are displayed under the 
main panel. The following parameters are shown by 
default: name, length (for common files) or type (for  
the rest of the files), access modes in the same format, 
as in the command ``{\tt ls~-l}'' (see ls~(1)).

Lines below the panels are intended for input commands. The
name of the current directory is used as a prompt.

The bottom line has an obvious hint---the names of
commands which are executed by {\tt F1}$\ldots${\tt F10}.

\subsection{Moving across Directories}

To move in the current directory, use arrow keys. The 
indicator of the current file moves in the panel and, 
reaching the panel border, scrolls a list of files
in the required direction. You can scroll many
files at once using PgUp and PgDn.

{\tt <Tab>} ({\tt \^{}I}) is used to go from the left panel
to the right one and vice versa.

To enter another directory, select its name on the 
screen and press {\tt <Enter>}. Name ``{\tt ..}'' shows a previous
directory in the tree, so you can reach any directory  of 
the file system. 

There are many other ways of moving in the file 
system, which will be discussed below (chapter ``Moving in File System'').

\subsection{Answering Inquiries}

DECO under operation can make inquiries. It can ask 
the name of the directory, to switch over to under the 
command ~{\tt \^{}R}, the name of the file or directory under
writing by the command of copying or renaming, etc. 
You have to know how to answer these inquiries. We shall
discuss, for example, the command ~{\tt \^{}R}---change of the directory name.

\fig{p5}{dirname}{Asking for Directory Name}

After~{\tt \^{}R} is received, DECO asks for directory name
(fig.~\ref{dirname}). The cursor stops at the directory name field.
There is a default for the given command in this field. It is the
name of the current directory. A user has some alternatives:
\begin{itemize}
\item
To cancel the enquiry and the command by pressing {\tt F10}.
\item
To press {\tt <Enter>}. In this case the default is used. It
is of little sense for {\tt \^{}R}, but such commands, as
the command of file copying, usually use the default.
\item
To input a new name. If the cursor was placed at the beginning of the line
(as, for example, in copying commands), then after the first
character has been put in the text to the right of the cursor will disappear.
\item
To edit the name given by default.
\end{itemize}

The following editing commands are available:
\begin{example}
\litem{$\leftarrow$, $\rightarrow$}
Moving along the line under editing.
\litem{{\tt <Tab>}}
Going to the beginning or end of the name.
\litem{{\tt <BackSpace>}}
Deleting the character to the left of the cursor.
\litem{{\tt \^{}Y}}
Deleting the line under editing.
\litem{{\tt <Del>} or {\tt \^{}G}}
Deleting the character just below the cursor.
\end{example}
Any other character is inserted, by moving the text apart.

\subsection{Error Diagnostics}

\fig{p6}{error}{Error Message}

In case of error, DECO outputs an enquiry with an
error diagnostics and proposes to confirm that it has 
been noticed. To proceed, press any key. An example 
of the output is shown in fig.~\ref{error}.

\subsection{Using Menu}

Two types of DECO commands have been discussed above:
commands started by functional keys (e.g. {\tt F1}), and commands started by
control-characters (e.g. {\tt \^{}R}---change of
directory). The third way to start a command is to use
the system menu. To do this, press {\tt F9}. In the upper
part of the screen a line with possible alternatives will 
appear (fig.~\ref{menu}).
 
\fig{p7}{menu}{Main Menu}

Titles ``Left'', ``Files'', etc. mean possible alternatives of the menu.
You can choose the menu you need by using the left-
or right-arrow keys. Each menu consists of
commands. Special codes are shown to the right of 
some commands, using which you can start these 
commands without starting the menu. Any DECO 
command can be called for through the menu, so that 
the menu is an addition to the hint. Stars may be shown to
the left of some commands which control setting up modes.
A star indicates that the mode is on. You can
call for a command in the menu by using the up- and down-arrow keys
and pressing {\tt <Enter>}. You can cancel the menu by pressing {\tt F10}.

After executing the command, DECO exits from the menu.
If you want to execute a command, staying in the menu,
you should press~{\tt \^{}J}.

Now we call for the screen redrawing command by using
the menu. First, press the left-arrow key twice, going to
the ``Commands'' menu. Then select the command of
screen redrawing (fig.~\ref{redraw}), press {\tt <Enter>}, and the
command will be executed.

\fig{p8}{redraw}{Redraw of Screen}

\section{Main DECO Capabilities}

\subsection{Choosing File Systems and Files}

\subsubsection{Moving along File System}

DECO is intended to simplify moving along directories. 

One of the files in the current panel (see fig.~\ref{screen}) is
painted or inverted---this is the current file. (In terminals
without inversion, square brackets are used).

Arrow keys make it possible to move in the panel and choose any
directory file. In large directories, you can use the 
{\tt <PgUp>} and {\tt <PgDn>} keys to move up or down one panel full
at a time or {\tt <Home>} and {\tt <End>} keys to go to the
beginning or end of the directory.

To select another directory, choose the directory name 
by the cursor and press {\tt <Enter>}. For instance, select
the file ``{\tt ..}'' and press {\tt <Enter>}. This file refers to the
parent's directory, so we shall find ourselves in the 
{\tt /usr/user} directory, and the directory where we have
just been (i.e. {\tt w}) will be the current one (see fig.~\ref{dirw}). To
return to the {\tt w} directory, press {\tt <Enter>} once again.

\fig{p9}{dirw}{Directory {\tt w}}

To change the current directory, you can also press {\tt \^{}R}.
The system will ask the name (fig.~\ref{dirname}).
You can press (see ``Answering Inquiries''):
\begin{example}
\litem{{\tt \^{}Y}}
Delete the line.
\litem{{\tt /}}
Choose directory name.
\litem{{\tt <Enter>}}
DECO will go to the root file system (fig.~\ref{root})
To return to your home directory, go step-by-step to
{\tt usr}, {\tt user}, {\tt w}.
\end{example}

\fig{p10}{root}{Root Directory}

To choose another directory, you can also execute the command {\tt cd}:
\begin{source}
\begin{verbatim}
cd directory-name
\end{verbatim}
\end{source}
or
\begin{source}
\begin{verbatim}
cd
\end{verbatim}
\end{source}
If the ``directory name'' is not given, the home directory
is implied. 

Type {\tt cd} (and press {\tt <Enter>}). DECO will go to the home
directory. The display will appear as in fig.~\ref{screen}.

Another possibility is to go from the left panel to the
right one and vice versa. The DECO current directory is 
being changed correspondingly. Use {\tt <Tab>} for this
purpose. You can swap panels by the {\tt \^{}U} command.

For example, try to output the command list in the {\tt /bin}
to the right panel.

\begin{example}
\litem{{\tt \^{}I}}
Go to the right panel.
\litem{{\tt \^{}R}}
Enquiry will appear (fig.~\ref{error}).
\litem{{\tt /bin}}
Input the directory name.
\litem{{\tt <Enter>}}
Execute transfer by pressing {\tt <Enter>}.
\litem{{\tt \^{}I}}
Return to the left panel.
\end{example}

The display is given in fig.~\ref{bin}.
 
\fig{p11}{bin}{Directory {\tt /bin}}

\subsubsection{Filename Search}

If the directory is large, it is rather difficult to scroll 
through it, to find the desired file. To do this, use the 
command of filename search: {\tt \^{}K}. We shall discuss now the
DECO state shown in fig.~\ref{bin}. We go to the {\tt /bin}
directory (by pressing {\tt \^{}I}) and try to find the {\tt cal}
command there. Press {\tt \^{}K}. The enquiry will appear (fig.~\ref{ca}).
Input the beginning of the filename you are looking for, that is ``{\tt ca}'',
and press {\tt <Enter>}. The cursor will be placed at the nearest file in
the directory, which has the name, beginning with {\tt ca} ({\tt cal} in
this case).

\fig{p12}{ca}{Fast File Search}

\subsubsection{File Selection}

DECO allows you to select files for further operations. If
you choose file {\tt tsimp.exe} and press {\tt <Ins>} (or {\tt \^{}T}),
the filename will be selected with `{\tt >}' and colored. The cursor moves
downwards, making it possible to select a number of successive files at once
without using arrows.

\fig{p13}{tagged}{Selected Files}

The files {\tt tsimp.exe}, {\tt tsimp.f}, {\tt tsimp.o} are selected in
fig.~\ref{tagged}. Repressing Ins above the selected file unselects it.

\fig{p14}{unselect}{Unselection}

Sometimes you have to select or unselect a few files at 
once. To do this, you have to press~`{\tt +}' at the beginning
of the command line to select, or press `{\tt -}' to unselect.
The enquiry ``Select the files'' or ``Unselect the files''
(fig.~\ref{unselect}) will appear.

You have to input a filename or a pattern of filename (see the next chapter),
and the files that suit this pattern will be selected (or unselected). The
pattern `{\tt *}' is used by default, it suits any name. Press `{\tt -}'.
Then input ``{\tt *.o}'' and press {\tt <Enter>}. DECO will unselect {\tt tsimp.o},
which suits the given pattern.

\subsubsection{Patterns}

Many DECO commands use a {\em name pattern}. Being compared with a file name,
some pattern characters are interpreted in a special way, quite like in
shell and cshell interpreters. They are as follows:
\begin{example}
\litem{{\tt *}}
Any number of arbitrary characters.
\litem{{\tt ?}}
One arbitrary symbol.
\litem{{\tt [ab0-9]}}
Any characters of the given set.
\litem{{\tt [\^{}ab0-9]}}
Any character, but the given ones.
\litem{{\tt \^{}}}
When used in the first position of the pattern,
it negates the meaning of the pattern.
\end{example}

For example, the ``{\tt *.o}'' pattern corresponds to {\tt tsimp.o},
{\tt fsplit.o} and does not correspond to {\tt tsimp.f}. The
``{\tt \^{}*.[ob]}'' pattern corresponds to any names, apart from
those ending in ``{\tt .o}'' and ``{\tt .b}''.

\subsection{Construction and Execution of Commands}

\subsubsection{File Executing}

The simplest way to execute a file under DECO is to choose this file
on the screen and press {\tt <Enter>}. If this file is executable, its name
will be transferred to the command line and command
will be executed. Otherwise, if the file is regular
then the command line will be built by some rules, discussed later.
After execution a prompt containing the directory name will appear.

For example, execute the command ``{\tt cal}''. This command is
located in the {\tt /bin} directory, which is shown on the
right side of fig.~\ref{bin}. After executing the previous
example this file is already selected; if not, select it by 
using the arrows and {\tt <PgUp>} and {\tt <PgDn>} keys.
Then press {\tt <Enter>}. The display is given in fig.~\ref{execution}.
 
\fig{p15}{execution}{Example of Command Execution}

To return to the screen mode, press {\tt <Enter>}.

\subsubsection{Entering and Executing Commands}

To execute a command under DECO, type in a UNIX command and press {\tt <Enter>}.
To start commands, DECO uses one of standard command interpreters, so you
can type in commands in the usual way using~``{\tt *}'', ``{\tt ?}'' and
other special symbols. However, you should remember that
the interpreter is restarted each time, so that variables 
and settings-up (aliases, keys, etc.) are not saved.

Under DECO you have additional possibilities. You can:
\begin{itemize}
\item
Edit the line, using keys:

\begin{example}
\litem{{\tt <BackSpace>}}
To cancel the character to the left of the cursor.
\litem{{\tt <Del>} or {\tt \^{}G}}
To cancel the character under the cursor.
\litem{{\tt \^{}A}}
To move along the line to the left.
\litem{{\tt \^{}D}}
To move along the line to the right.
\end{example}

\item
Insert the name of the current file (or the list of
the selected files, see below) in the line, by pressing {\tt \^{}J}.
\item
Insert one of the previous commands, by pressing {\tt \^{}E}.
\item
Go to the editing mode using arrow keys, by pressing {\tt \^{}P}.
Repressing {\tt \^{}P} will return DECO to the file selection mode.
In the line editing mode commands of moving along words are added:
the up-arrow key brings the cursor to the previous word, the down-arrow
key---to the next word.
\end{itemize}

To enter nontext symbol you can use the command~{\tt \^{}V}.  The next
following char will be inserted into the command line.

After executing the command, DECO does not return to
the screen mode immediately, but outputs its prompt in 
the common terminal operation mode without erasing 
the screen. The commands {\tt F2} (the user's menu), {\tt F7}
(creating a directory), {\tt F9} (the DECO menu), {\tt F10} (Exit)
are recognized. You can either input a new command, 
or return to the screen mode by pressing {\tt <Enter>}, or go
to the editing mode by pressing {\tt \^{}P}. In the latter case,
DECO does not draw the whole screen, but just the frames and the command
itself. Repressing {\tt \^{}P} returns DECO to the screen mode.

For example, if you find yourself as in fig.~\ref{bin}, do the following:
\begin{example}
\litem{{\tt wc\vs}}
Type command and space.
\litem{$\uparrow$ $\downarrow$ $\leftarrow$ $\rightarrow$}
Select a file (for example, {\tt tsimp.f}).
\litem{{\tt \^{}J}}
Insert the word {\tt tsimp.f} into the command line.
\litem{{\tt <Enter>}}
Execute the command ``{\tt wc tsimp.f}''.
\end{example}

The picture on the screen is shown in fig.~\ref{wc}.

\fig{p16}{wc}{Screen after Command Execution}

Enter a new command, without pressing {\tt <Enter>}:
\begin{source}
\begin{verbatim}
ww *.c
\end{verbatim}
\end{source}
We made an error and typed {\tt ww} instead of {\tt wc}. Press
{\tt \^{}P} to go to the editing mode (fig.~\ref{edlin}).

\fig{p17}{edlin}{Line Editing}

And now, edit the line and execute the command. Then 
press {\tt <Enter>} and return to the DECO screen (fig.~\ref{bin}).

Another alternative is to use the command history.

\subsubsection{Command Repeat and Work Protocol}

DECO stores commands, the user entered, and allows 
to repeat them. You can use the following DECO commands:

\begin{example}
\litem{{\tt \^{}E}}
To insert the previous command, i.e. it moves the protocol backwards.
\litem{{\tt \^{}X}}
To insert the corresponding command, i.e. it moves the protocol forwards.
\litem{{\tt \^{}B}}
To enter the protocol. This command displays the user's
protocol (fig.~\ref{history}).
\end{example}
 
\fig{p18}{history}{Command History}

After entering the protocol the following commands are available:
\begin{example}
\litem{$\uparrow$, $\downarrow$}
To select commands from the protocol.
\litem{{\tt <Enter>}}
To execute the selected command.
\litem{{\tt \^{}J}}
To insert the selected command and switch to the command input mode.
\litem{F10}
To quit the protocol.
\end{example}

For example, you want to repeat the command {\tt wc tsimp.f}
(fig.~\ref{history}). Select this command by the arrows and
press {\tt \^{}J}. The protocol will disappear from the screen
and {\tt wc tsimp.f} will appear in the command line.
Reenter the protocol {\tt \^{}B}, select the command, and
press {\tt <Enter>}. The command will be executed and
DECO will wait for the next command (fig.~\ref{wc}). To return
to the screen DECO mode, press {\tt <Enter>}.

\subsubsection{Built-in Commands}

Usually DECO executes command lines by calling the system command
interpreter.  But some commands, e.g. {\tt cd},
are executed by DECO itself, and are called built-in commands.
All built-in commands are listed in table~\ref{builtin}.

\begin{table}[bp]
\begin{center}
\begin{tabular}{|ll|}
\hline
{\tt cd} & \\
{\tt chdir} & Go to home directory \\
& \\
{\tt cd dirname} & \\
{\tt chdir dirname} & Go to specified directory \\
& \\
{\tt set name value} & \\
{\tt setenv name value} & Set value of global variable \\
& \\
{\tt unset name} & \\
{\tt unsetenv name} & Delete global variable \\
& \\
{\tt env} & \\
{\tt printenv} & Print list of global variables \\
& \\
{\tt pwd} & Print name of current directory \\
& \\
{\tt exit} & Quit DECO \\
\hline
\end{tabular}
\caption{Built-in Commands}
\label{builtin}
\end{center}
\end{table}

\subsection{File View and File Edit}

DECO makes it possible to view or edit the current file 
by {\tt F3} (View) and {\tt F4} (Edit). Built-in view and edit
programs are used by default. Using the Menu Mode 
(see ``Menu Mode'' below), you can set up other
programs for viewing and editing files (e.g. the
{\tt less} command to view and the {\tt vi} command to edit).

\subsubsection{File View}

We shall discuss fig.~\ref{bin} again. Select the {\tt tsimp.f} and
press {\tt F3}. The built-in view program will be started (fig.~\ref{view}).
 
\fig{p19}{view}{File View}

The top line contains the name of the file under viewing,
the size of the file in terms of lines and bytes.
The current line and the current shift of the screen
in terms of columns are indicated in the right upper corner.

You can move through the file using the up- and down-arrow
keys to scroll up or down one line at a time or
the left- or right-arrow keys to move 40 columns to the
left or to the right. Use {\tt <PgUp>} and {\tt <PgDn>} keys to
move up or down one panel full at a time or {\tt <Home>}
and {\tt <End>} keys to go to the beginning or end of the file.
Additional commands are input by functional keys, the 
list of them is given in the bottom line:

\fig{p21}{hexview}{Hexadecimal View}

\begin{example}
\litem{F3~--- Raw}
To switch the character highlighting mode with code over
200${}_8$. The characters are output in the 7-bit code and colored by default.
In the ``raw'' mode they are output in the 8-bit code to the terminal.
\litem{F4~--- Hex}
To switch viewing in the form of dump (fig.~\ref{hexview}).
Character codes are given on the left; the same characters in
the text form are output on the right. The cursor can be moved
from the text part of the screen to hex part and vice versa by {\tt <Tab>}.
\litem{F5~--- Top}
To go to the beginning of the file.
\litem{F6~--- Bottom}
To go to the end of the file.
\litem{F7~--- Search}
To search. You are asked for the text to search (fig.~\ref{search}).
The text will be searched starting from the first line. To continue the search,
press {\tt F7} and then press {\tt <Enter>} to answer the enquiry.
There is an additional facility in the Hex mode: you can indicate arbitrary
characters in the search string, the hex code being set as {\tt \bs xx}.
\litem{F9~--- Spaces}
To switch the space and tabulation displaying mode. Spaces are
designated as dim points, and tabulations---as dim characters of underlining.
\litem{F10~--- Quit}
To quit the viewer.
\end{example}

\fig{p22}{search}{Text Search}

Control characters with codes from O to 31 are given as characters with codes
increased by 64, but of decreased brightness (control-characters).
Characters with codes from 128 to 196 are inverted and bright.

\fig{p23}{edit}{Built-in Editor}

\subsubsection{File Edit}

To invoke the built-in editor, press {\tt F4} (fig.~\ref{edit}).
The screen is similar to that of the viewer. The field ``Char''
in the right upper corner displays the current character code.
The following commands are available in the built-in editor:

\begin{example}
\litem{$\uparrow$ $\downarrow$ $\leftarrow$ $\rightarrow$}
Moves the cursor through the file.
\litem{{\tt <BackSpace>}}
Serves to delete the character under the cursor, just like in line editing.
\litem{{\tt <Del>}}
Serves to delete the character to the left of the cursor.
\litem{{\tt \^{}Y}}
Deletes the current line.
\litem{{\tt \^{}K}}
Deletes the text starting from the cursor till the end of the line.
\litem{{\tt \^{}V}}
Inserts the next character into text. This command is used for
putting non-ascii symbols into text.
\litem{F2~--- Save}
Saves the file. 
\litem{F3~--- Raw}
Functions as {\tt F3} in the view command: it switches the displaying
character mode with codes greater than 0200${}_8$.
\litem{F5~--- Top}
Transfers to the beginning of the file.
\litem{F6~--- Bottom}
Transfers to the end of the file.
\litem{F7~--- Search}
Searches by image, quite like the viewer.
\litem{F8~--- Home}
Transfers to the beginning of the screen.
\litem{F9~--- Spaces}
Switches the space and tabulation displaying mode. Spaces are
designated by dim points, and tabulations---by dim underlines.
\litem{F10~--- Quit}
Exits the editor. If the text has been changed anyhow, you are asked,
whether to save changes, or not to save, or to continue editing
(fig.~\ref{save}).
\end{example}

\fig{p24}{save}{Exit from Editor}

To insert a new line before the current one, you have to 
press {\tt <Enter>} at the beginning of the current line. In
any other location, {\tt <Enter>} will cut the line in two parts.
To reconnect the lines, input {\tt <Backspace>} at the first
line position.

The built-in editor is mainly intended to update small 
files. This editor has a limited size of the file under 
editing and a limited command set, but is simple in 
exploring and using. The user can set the start of 
another editor by {\tt F4} using the Setup Mode.

\subsubsection{File Creating}

To create a new file, just select the ``{\tt .}'' directory and
press {\tt F4}. The system will ask the filename and then
create an empty file under this filename and start the editor.

\subsection{File Operations}

File operations started by the keys {\tt F5}$\ldots${\tt F8} are described
below (copying, renaming, deleting, creating directories). All these
operations with the exception of directory creating, can be used with the
current file or with selected files. See ``File Selection''
about selected files.

\subsubsection{Copying and Moving Files}

{\tt F5} starts copying files, {\tt F6} initiates file renaming. These
commands copy or move the current file or the selected ones to the directory,
the name of which is given by the user. By default, it is the name of
the second (not current) panel. If one file is being copied or renamed,
you can give a new filename instead of the name of the directory.

Example. We shall return to the first example (fig.~\ref{screen}) and copy
the file {\tt tsimp.exe} to the home directory. To do this, select the file
{\tt tsimp.exe} in the left panel using arrow keys and press {\tt F5}
(fig.~\ref{copy}).

\fig{p25}{copy}{File Copy}

DECO asks you either to confirm (by pressing {\tt <Enter>}) that you want to copy
the file to the right panel (the {\tt /usr/user} directory) or to use
another name for copying. You have three alternatives:

\begin{itemize}
\item
Press {\tt <Enter>}. Copying starts and the corresponding message appears on the
screen. When the file copying is over, the message will disappear.
\item
Input a new directory name. After the first character is input, the string
in the enquiry is deleted and the name is input instead of the name given
in the enquiry.
\item
Edit the proposed name (see section ``Answering Enquiries'').
\end{itemize}

{\tt F6} functions in the same manner but the file is not
copied---it is renamed if it is possible, or rewritten to a
new location and erased in the previous one (the {\tt mv}
command is used instead of {\tt cp}).

As an example, copy all the files from the {\tt doc} directory
to the {\tt w} directory. To do this, you have to enter:
\begin{example}
\litem{{\tt \^{}I}}
To go to the right panel.
\litem{{\tt cd doc}}
To go to the {\tt doc} directory.
\litem{{\tt + <Enter>}}
To choose all the regular files in this directory.
\litem{F5}
To start copying. The name enquiry appears.
\litem{{\tt <Enter>}}
To confirm the name.
\end{example}

As a result, all the right panel files will appear in the left
panel. Remove them back now. To do this, you should type:

\begin{example}
\litem{{\tt \^{}I}}
To return to the left panel.
\litem{{\tt +}}
To ask for the file selection.
\litem{{\tt \^{}Y} *.doc {\tt <Enter>}}
To select files with documentation.
\litem{F6}
To call the rename command. The name is asked for.
\litem{{\tt <Enter>}}
To confirm that you want to remove four files from the current
directory to the {\tt /usr/user/doc}.
\end{example}

There are similar files, however, in the right directory. So,
the system outputs an enquiry, as in fig.~\ref{move}.

\fig{p26}{move}{File Move}

You have a choice---whether to place the indicated file
onto the location of the existing file (by default), or to 
write all the files without any questions, or not to 
remove the indicated file. Choose ``All'', press {\tt <Enter>},
and the selected files are removed to the {\tt doc} directory.

\subsubsection{Directory Creating}

{\tt F7} creates a new directory in the current directory. The
user is asked for the name. 

\subsubsection{File Deleting}

{\tt F8} allows you to execute one of the most important
operations--- to delete files and directories. If no files are
selected, the current file or directory is deleted (only 
empty directories can be deleted). If any files are 
selected, all of them are deleted. In any case, you are 
asked for confirmation. 

Example. Create the directory {\tt /usr/user/w/tst} and
rewrite all the documents from the {\tt /usr/user/doc} there.
The {\tt doc} is displayed in the right panel,
{\tt w} is displayed in the left one (after the last example):

\begin{example}
\litem{F7}
Creates the directory.
\litem{{\tt tst <Enter>}}
Names the directory.
\litem{{\tt cd tst}}
Changes the current directory to {\tt tst}.
\litem{{\tt \^{}I}}
Switches to another panel.
\litem{{\tt + * <Enter>}}
Selects all the files in {\tt doc}.
\litem{F5}
Initiates copying.
\litem{{\tt <Enter>}}
Confirms copying.
\end{example}

Now, go to the {\tt tst} directory. Select all the files there and
delete them. To do this, press: 

\begin{example}
\litem{{\tt \^{}I}}
To go to the left.
\litem{{\tt + <Enter>}}
To select all the files.
\litem{F8}
To start deleting.
\end{example}

\fig{p27}{delete}{File Deleting}

The system will ask you twice. First, it asks you if you really have
selected four files, and second, if you really want to delete
them (fig.~\ref{delete}). If you confirm it, the selected files will be
erased. The directory is still to be deleted:

\begin{example}
\litem{{\tt <Home>}}
Choose directory ``{\tt ..}''.
\litem{{\tt <Enter>}}
Return to {\tt /usr/user}. The {\tt tst} directory is now the current one.
\litem{F8}
Delete {\tt tst}.
\end{example}

The system will ask you if you really wish to delete the
directory {\tt tst}. Just press {\tt <Enter>} to delete the directory.

If after receiving the enquiry you find out that you have 
made a mistake and nothing needs deleting, just press {\tt F10}, or
select ``Cancel'', and your erroneous command will be aborted.

\section{DECO Extended Functions}

DECO basic functions, which are intended to be widely used by every user
are described above. DECO extended functions are described below, that
make it possible to accommodate DECO to the needs of individual users,
as well as some rarely used commands which can be called for through the DECO
system menu.

\subsection{File Execution. Default Rules}

If you press {\tt <Enter>} while running DECO and the command line is empty,
the following occurs:
\begin{itemize}
\item
If the file is a directory, the system goes to this directory.
\item
If the file is a regular one and is executable (`x'
as an attribute of the file), this file is executed.
\item
If this file is a regular one and is not executable, the command that
fits its name is executed. The correspondence is described in
the file {\tt \~{}/.deco} (`{\tt \~{}}' means the user's home directory).
This allows the user to execute default actions for each file, which
depends upon its name. For example, you can start {\tt make} for
{\tt Makefile}, call the editor for files with the extension ``{\tt .c}'',
etc. If the file {\tt \~{}/.deco} is not found, DECO searches descriptions
in the file {\tt /usr/local/lib/deco/profile} and then in
{\tt /usr/lib/deco/profile}.
\end{itemize}

The file which contains descriptions of defaults has the following structure:
\begin{itemize}
\item
The strings, beginning with ``{\tt \#}'', are comments.
\item
Each action can be described by a couple of lines:
\begin{source}
\begin{verbatim}
<pattern1> <pattern2>...<patternN>
       <command>
\end{verbatim}
\end{source}
The patterns begin from the first column, and follow the pattern
matching rules, which are described in section ``Patterns''.
\end{itemize}

\subsubsection{Macro in Description of Defaults and in Menu}

Commands in the default description file, as well as in the user's menu, may
contain {\em macro} of form {\tt \%X}. Before executing the command
macros are replaced by some parameters. The full list of available macros
and their meanings is given in table~\ref{macro}.

\begin{table}[bp]
\begin{center}
\begin{tabular}{|cl|}
\hline
{\tt \%f} & name of current file \\
{\tt \%b} & name of current file without extensions \\
{\tt \%t} & list of selected files \\
{\tt \%d} & name of current directory (last component) \\
{\tt \%c} & full name of current directory \\
{\tt \%u} & name of user \\
{\tt \%g} & name of group of users \\
{\tt \%\%} & character `{\tt \%}' \\
\hline
\end{tabular}
\caption{Macro in Commands}
\label{macro}
\end{center}
\end{table}

The following characters may be placed at the beginning of a command:

\begin{example}
\litem{{\tt -}}
To remain in the screen mode after execution.
\litem{{\tt @}}
Not to insert the command into the history.
\end{example}

\subsubsection{Example of File {\tt \~{}/.deco}}

Example of file \~{}/.deco:
\begin{source}
\begin{verbatim}
# bak files - delete
*.b
       rm %f

# Makefile - run ``make''
Makefile makefile *.mk
       make -f %f

# core - call debugger
core
       adb

# object file - show name list ``nm''
*.o
       nm %f | more

# archive file - show list of contents
*.a
       ar tv %f | more

# doc - format it
*.me
       nroff -me %f > %b.lp

# manual - format
*.man
       nroff -man %f > %b.lp
\end{verbatim}
\end{source}

\subsection{User's Menu}

A DECO user can create his own menu and then call
for it by {\tt F2}. It commands the system to look for the
description of the menu in the file menu (in the current 
directory), and if there is no description there---in {\tt \~{}/.menu},
{\tt /usr/local/lib/deco/menu} and in {\tt /usr/lib/deco/menu}. A
file with the description of the menu has the following structure:
\begin{itemize}
\item
Strings beginning with `{\tt \#}', or empty ones, are considered as comments
and ignored.
\item
The description of menu line takes two lines in the file. Its format is:
\begin{source}
\begin{verbatim}
<character> <description>
       <command>
\end{verbatim}
\end{source}
\end{itemize}
where:
\begin{example}
\litem{$<$character$>$}
Either a single character A$\ldots$Z, a$\ldots$z, 0$\ldots$9
(typed at the first position of the line), pressing which you start the given
command; or a two-letter name of the key {\tt F1}$\ldots${\tt F9}.
\litem{$<$description$>$}
The text that is output after this character. This text serves as user's
prompt.
\litem{$<$command$>$}
UNIX command(s) that will be executed when the indicated line of the menu is
selected. The command is executed by the {\tt /bin/sh} or {\tt /bin/csh}
interpreter (depending on the settings-up). It may include macro, such as
{\tt \%X} (see the previous chapter). The command text is not written from
the first position, spaces before the command are ignored. Special characters,
such as `{\tt -}' or `{\tt @}' (see the previous chapter), can also be
indicated before the command.
\end{example}

In the menu mode, the ``!file-name'' built-in command is recognized,
by which a new menu can be called for from the specified file.

We shall discuss now an example of operating with the user's menu. Return to
fig.~\ref{screen} (DECO initial state) and create three files in the home
directory. Go to the right panel ({\tt \^{}I}), choose directory ``{\tt .}'',
press {\tt F4}, enter file name {\tt .menu}, then type the text,
given below. By analogy, create two other files. Then return to the left
panel---the {\tt w} directory.

The menu files for the given example are:

File {\tt .menu}:
\begin{source}
\begin{verbatim}
#
# Example of the menu with inserted starts
#
F1 Floppy disk management
       !%h/.menuf

F2 Remote logins
       !%h/.menur

F3 Get mail
       -@mail
\end{verbatim}
\end{source}

File {\tt .menuf}:
\begin{source}
\begin{verbatim}
#
# Menu for operations with floppy disks
#
F      Format floppy as 1.2 MB
       -@format /dev/rfd096ds15

f      Format floppy as 360 k
       -@format /dev/rfd048ds9

c      Write current dir/file to floppy
       tar cvf /dev/rinstall %f

M      Mount floppy to /mnt
       /etc/mount /dev/install /mnt

U      Unmount floppy
       /etc/umount /dev/install

L      List mounted file systems
       /etc/mount
\end{verbatim}
\end{source}

File {\tt .menur}:
\begin{source}
\begin{verbatim}
#
# Menu for remote input to computers
#
a      Rlogin amet
       -@rlogin amet

b      Rlogin berta
       -@rlogin berta

c      Rlogin cospas
       -@rlogin cospas
\end{verbatim}
\end{source}

Now {\tt w} is the current directory (fig.~\ref{w}). Press {\tt F2}. First,
DECO will look for {\tt .menu} in the current directory, then
(as there is no {\tt .menu} in the current directory) in the
home directory (the right panel in fig.~\ref{w}).
 
\fig{p28}{w}{Menu Files}

\fig{p29}{fmenu}{User Menu}

In accordance with the file {\tt \~{}/.menu}, the screen will be as shown
in fig.~\ref{fmenu}. On the left of the screen you see the keys, on the
right---system responses to pressing these keys (strictly speaking, it
is the description text from the {\tt .menu} file). The arrow to the right of
the key name indicates that another menu is called for in the given line.
If you want, for example, to call for the floppy managing menu, you have
either to press {\tt F1} (as is on the left), or just to select the required
line by using the up- and down-arrow keys and pressing {\tt <Enter>}. DECO will
choose the line with the command {\tt !\%h/.menuf} and expand ``{\tt \%h}''.
As a result, you will get the command {\tt !/usr/user/.menuf}.
The format menu appears on the screen (fig.~\ref{fmenuf}).

\fig{p30}{fmenuf}{Format Menu}

To output the list of mounted file systems (the bottom line in the menu in
fig.~\ref{fmenuf}), press `{\tt L}' and get the mount command. As there is
neither `{\tt -}', nor `{\tt @}' before this command, it will be
executed, as if it had been input from the keyboard, i.e. it will be
added to the DECO history, while DECO will remain in the non-screen
operation mode. Press {\tt <Enter>} to return to the screen mode (fig.~\ref{w}).

\subsection{DECO Main Menu}

All DECO commands are included in the main menu that you can start by
{\tt F9}. Besides, commands that are often used can be called for by
functional and control keys. Some of these commands are discussed above.
We shall discuss the commands of the main menu one by one, especially those
commands that have not been considered yet.

Press {\tt F9}. The main menu appears on the screen (fig.~\ref{menu}).
By using the left- and right-arrow keys you can move along the upper screen
line and choose one of such menu items as ``Left'', ``Files'', ``Commands'',
``Options'', ``Right''. The corresponding pop-up menu appears on the
screen. The up- and down-arrow keys serve to choose commands from these menus.
To execute the chosen command, you should press {\tt <Enter>}; to quit
the pop-up menu, press {\tt F10}. An alternative way of executing some commands
by using control symbols is shown to the right of these commands. For
example, ``Full screen~--- {\tt \^{}F}'' means that the command can be
called for either by selecting it in the menu, or by pressing {\tt \^{}F} in the
DECO normal operation mode (outside the menu).

Now we shall consider all groups of commands.

\subsubsection{Options Menu}

You have to learn this menu first, because it includes  
a number of necessary DECO settings-up. See fig.~\ref{options}.
 
\fig{p31}{options}{Options Menu}

Commands of Options Menu:
\begin{example}
\litem{Switch panels~--- {\tt \^{}I}}
Going from one panel to another.
\litem{Swap panels~--- {\tt \^{}U}}
Swaps the right and left panels.
\litem{Full Screen~--- {\tt \^{}F}}
Widely used command. Allows to change the size of the panel. In the normal
mode panels usually occupy about half of the screen, and the other half is
used for typing commands. Retyping the command resets the system. In the full
screen mode the panel occupies as large a place as possible on the screen,
leaving one line for the command field.
\litem{Double width~--- {\tt \^{}W}}
Double width of the screen. This command allows you to extend the
current panel to the size of the whole screen, by deleting the second panel
from the screen. This command is also widely used for operating with large
directories. Retyping the command resets the system. Fig.~\ref{dwidth}
shows the DECO screen, when both modes---full screen and double width---are
operating. 

\fig{p32}{dwidth}{Double Width of Window}

\litem{Command line mode~--- {\tt \^{}P}}
This command was described above (see ``Command Constructing and Executing'').
It allows you to switch from moving inside the panels to the mode of
editing the command line and vice versa.
\litem{Show hidden files}
This command switches the mode of displaying files, whose names begin with dot.

\fig{p33}{viewer}{Viewer Prompt}

\fig{p34}{extview}{Setup of External Viewer}

\litem{Viewer$\ldots$}
Sets up the file viewing program, which is called for by {\tt F3}.
The system asks you a question (see fig.~\ref{viewer}). ``Built in''
means using the built-in viewing program.
If you choose ``External'' and press Enter, the system will ask you the name
of the viewing program (fig.~\ref{extview}). If you, for instance, input
{\tt less}, then the UNIX command {\tt less} will be called for by pressing
{\tt F3} to view files.
\litem{Editor$\ldots$}
Sets up the editing command. The same as the previous command. The built-in
editor is used by default. 

\fig{p35}{shell}{Setup of Shell}

\litem{Shell$\ldots$}
Sets up the command interpreter, by means of which DECO executes UNIX
commands (fig.~\ref{shell}). DECO can use either {\tt /bin/sh} (``Shell''
field) or {\tt /bin/csh} (``Cshell'' field) command interpreters. Shell is
used by default. The difference between these two command interpreters can be
illustrated, for example, by the command ``{\tt echo \~{}}'': in response to
this command Shell will output `{\tt \~{}}', while Cshell will give
the name of the user's home directory.
\litem{Save setup}
DECO current modes are saved in the {\tt \~{}/.decoini} file.
While starting, DECO looks for the initial settings, first,
in the file {\tt \~{}/.decoini}, then in {\tt /usr/local/lib/initfile} and in
{\tt /usr/lib/initfile} and installs the modes that are indicated there.
\end{example}

\subsubsection{Managing Panels}
 
The ``Left'' and ``Right'' menus allow you to set up the format of the left and
right panels. The ``Left'' menu is shown in fig.~\ref{menu}, The ``Right''
menu is the same, but refers to the right panel. The following commands are
recognized here:
\begin{example}
\litem{Brief, Long, Full}
Are modes of displaying files on the screen. The current mode is marked
with an asterisk. In the brief mode, only names and types of files are
displayed (see all the previous examples, beginning from fig.~\ref{screen}).
In the long mode you can see file lengths and access attributes
(fig.~\ref{long}).
\end{example}

\fig{p37}{long}{Long Mode}

The full mode turns on the double width mode (see the previous chapter)
and shows all the information about the file (fig.~\ref{full}).

\fig{p38}{full}{Full Mode}

In the full mode file owner and group names, file modification date and time
are also displayed. In the bottom line of the panel are also shown: device
and i-node numbers, link count, time and date of last access, time and date
of status change.

\begin{example}
\litem{Status~--- {\tt \^{}L}}
Switches the displaying of the status of the current directory.
Status is displayed in the opposite panel (fig.~\ref{status}).
\end{example}

\fig{p39}{status}{Status Panel}

In the status panel are displayed: directory name, number of files in the
directory and total size, and also number and size of selected files.

\begin{example}
\litem{Align extensions}
Controls separating file name extensions to the right margin.
\litem{Sort by name}
Sets sorting by file names.
\litem{Extension}
Sets sorting by file extensions.
\litem{Time}
Sets sorting by modification time.
\litem{Size}
Sets sorting by file size.
\litem{Unsorted}
Turns off file sorting by parameter. Sorting by file type is not changed.
\litem{Reverse}
Sets reverse sorting.
\litem{Sort by type}
Turns off file sorting by type. Sorting by parameter is not changed.
\litem{Re-read~--- {\tt \^{}R}}
Changes the current directory. You can call this command by typing~{\tt \^{}R}.
\litem{Pattern}
Sets {\em pattern of visibility} for files. The panel will display
only those files, names of which match the specified pattern.
Matching rules are described in section ``Patterns''.
\end{example}

Example. Let's set the pattern ``{\tt \^{}*.o}'' in the left panel
in fig.~\ref{status}. This pattern means ``all, that doesn't end
in `{\tt .o}'\,''. The result is shown in fig.~\ref{pattern}.

\fig{p40}{pattern}{Setting File Name Pattern}

\subsubsection{Files Menu}

The Files menu is given in fig.~\ref{files}. All operations are
described above (see ``File Operations'').
 
\begin{example}
\litem{Help~--- F1}
Shows a brief description of all commands.
\litem{User Menu~--- F2}
Calls for user's menu.
\litem{View~--- F3}
Views the file.
\litem{Edit~--- F4}
Edits the file.
\litem{Copy~--- F5}
Copies files.
\litem{Link}
Links files. It is similar to ``Copy'', but uses {\tt ln}
instead of {\tt cp} (see. ln(1)).
\litem{Symlink}
Creates symbolic links to files. Uses the ``{\tt ln -s}'' command (see. ln(1)).
\litem{Rename/move~--- F6}
Renames files or directories.
\litem{Make directory~--- F7}
Creates a directory.
\litem{Delete~--- F8}
Deletes files or directories.
\litem{Select Group~--- {\tt +}}
Selects a group of files by pattern.
\litem{Unselect Group~--- {\tt -}}
Unselects a group of files.
\litem{Quit~--- F10}
Quits DECO.
\end{example}

\fig{p41}{files}{Files Menu}

\subsubsection{Commands Menu}

The Commands menu (fig.~\ref{commands}) has a number of
commands, not included in other menus.
 
\fig{p42}{commands}{Commands Menu}

\begin{example}
\litem{Run subshell}
This command runs {\tt /bin/sh} (or {\tt /bin/csh}) as a subprocess.
After you exit from this command interpreter, you can
continue to work in DECO.
\litem{Compare directories}
This command compares the directories in the right and left panels. For each
panel, files are selected, which have no exact copy in the opposite panel. For
instance, fig.~\ref{cmpdir} shows the result of comparing {\tt /usr/user}
and {\tt /usr/user/w} directories which were displayed in fig.~\ref{w}.

\fig{p43}{cmpdir}{Comparing Directories}

\litem{Find file~--- {\tt \^{}K}}
Selects the file by using its name (see section ``Filename Search'').
\litem{History~--- {\tt \^{}B}}
Operates with the history (see section ``Command Repeat and Work Protocol'').
\litem{Home directory~--- {\tt \^{}{[}}}
Goes to the home directory (``{\tt cd}''). It can also be called for as
{\tt \^{}{[}}.
\litem{Root directory~--- {\tt \^{}\^{}}}
The same as the previous command, but goes to the root directory
(``{\tt cd /}'').
\litem{Redraw screen~--- {\tt \^{}{]}}}
Redraw the screen.
\litem{Menu file edit}
Starts editing user's menu file.
\litem{Extensions file edit}
Starts editing extension file.
\end{example}

\end{document}
